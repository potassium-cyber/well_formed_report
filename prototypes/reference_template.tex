\documentclass[12pt,a4paper]{article}
\usepackage{graphicx}
\usepackage{ctex}     % 支持中文
\usepackage{geometry}
\usepackage{fontspec} % 使用本地字体(如果用Overleaf记得设置路径)
\usepackage{pgffor}   % foreach宏包(如果需要处理分字)
\usepackage{ragged2e}

\geometry{top=2.5cm,bottom=2.5cm,left=3cm,right=3cm}

% ------- 定义基本变量 -------
\newcommand{\studentname}{张三}
\newcommand{\englstudentname}{San Zhang}
\newcommand{\studentid}{20221001}
\newcommand{\major}{应用物理专业}
\newcommand{\college}{物理科学与技术学院}
\newcommand{\supervisor}{车潇洋}
\newcommand{\englsupervisor}{Xiaoyang Che}
\newcommand{\papertitle}{科学前沿专题讲座}
\newcommand{\papertitleEN}{Course Paper for Special Topics on Scientific Frontiers}

% 封面用横线填写区
\newcommand{\fillin}[1]{\underline{\makebox[7cm][l]{\hspace{0.5cm}#1}}}
\newcommand{\fillinshort}[1]{\underline{\makebox[1cm][c]{#1}}}

% 华文行楷字体(如果Overleaf需要自行上传字体或换FandolKai)
\newcommand{\xingkai}[1]{{\CJKfontspec{STXingkai} #1}}

% ------- 正文开始 -------
\begin{document}

% 封面
\begin{titlepage}
    \centering
    \includegraphics[width=10cm]{school_logo.png}

    \vspace{2.5cm}

    \zihao{-0}
    {\xingkai{\textbf{《\papertitle》}}}\\[1cm]
    {\xingkai{\textbf{课程论文}}}\\[3cm]

    \zihao{-2}
    \begin{tabular}{p{5.3cm}p{10cm}}
        \raggedleft \textbf{学\qquad 院}: & \fillin{\college} \\[0.4cm]
        \raggedleft \textbf{专\qquad 业}: & \fillin{\major} \\[0.4cm]
        \raggedleft \textbf{年\qquad 级}: & \fillin{2022级} \\[0.4cm]
        \raggedleft \textbf{学\qquad 号}: & \fillin{\studentid} \\[0.4cm]
        \raggedleft \textbf{姓\qquad 名}: & \fillin{\studentname} \\[0.4cm]
        \raggedleft \textbf{指导老师}: & \fillin{\supervisor} \\
    \end{tabular}

    \vfill

    \zihao{4}
    完成日期:\quad \fillinshort{2025}\quad 年\quad \fillinshort{5}\quad 月
\end{titlepage}

% 扉页(中文英文摘要)
\newpage
\pagenumbering{roman} % 开始用小写罗马数字(i, ii, iii)

\centering
\zihao{3}\textbf{\papertitle}\\[0.5cm]
\vspace{0.5cm}
\zihao{4}
\studentname\qquad 指导老师:\supervisor\\[1cm]

\noindent
\zihao{3}\textbf{摘\quad 要}\\[0.5cm]

\justifying % ⭐强制两端对齐
\noindent
\zihao{-4}
根据国内外流域生态韧性研究,基于PSR框架建立漓江流域生态韧性评估框架。运用熵值法计算得到2005-2010年的生态环境状况,压力指数(P)占据0.248,状态指数(S)占据0.431,响应指数(R)占据0.321;其中正向指标权重之和为0.554,负向指标权重之和为0.446。结果表明,漓江流域生态环境质量显现出恢复状态。这主要是改善生活质量的响应措施不断加强,以及状态指标中的工业固体废物综合利用量对资源的循环利用,更好的改善了生态环境对的质量。\\[1cm]

\noindent
\textbf{关键词:关键词1;关键词2;关键词3;关键词4}

\newpage
\centering
\zihao{3}\textbf{\papertitleEN}\\[0.5cm]
\vspace{0.5cm}
\zihao{4}
\englstudentname\qquad Supervisor: Dr. \englsupervisor\\[1cm]

\noindent
\zihao{3}\textbf{Abstract}\\[0.5cm]

\justifying % ⭐强制两端对齐
\noindent
\zihao{-4}
Based on domestic and international research on watershed ecological resilience, this study established an ecological resilience evaluation framework for the Lijiang River Basin using the Pressure-State-Response (PSR) framework. By applying the entropy method to assess ecological conditions from 2005 to 2010, the results revealed the following indices: pressure index (P) accounted for 0.248, state index (S) for 0.431, and response index (R) for 0.321. Notably, the cumulative weight of positive indicators (0.554) exceeded that of negative indicators (0.446), indicating a recovery trend in the ecological environment quality of the Lijiang River Basin. This improvement is attributed to the continuous enhancement of response measures aimed at improving quality of life, particularly the comprehensive utilization of industrial solid waste under state indicators, which promoted resource recycling and significantly improved ecological quality. The findings underscore the effectiveness of targeted management strategies in fostering ecological restoration and sustainable development in the basin.\\[1cm]

\noindent
\textbf{Keywords: keyword1; keyword2; keyword3; keyword4}

% 目录页
\newpage
\centering
\vspace{1cm}
\tableofcontents

% 正文开始
\newpage
\pagenumbering{arabic}
\setcounter{page}{1}

\justifying % ⭐正文也强制两端对齐

% 正文示范起步
\section{引言}
这里是引言部分内容。\\

\section{研究背景}
这里是研究背景的内容。\\

\section{方法与材料}
\subsection{数据采集}
这里是数据采集的方法描述。\\

\subsection{数据处理}
这里是数据处理的方法描述。\\

\section{结果与讨论}
这里是结果与讨论的内容。\\

\section{结论}
这里是结论的内容。\\

\end{document}