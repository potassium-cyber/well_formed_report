\documentclass[12pt,a4paper]{article}
\usepackage{graphicx}
\usepackage{geometry}
\usepackage{fontspec}
\usepackage{xeCJK}     % 显式使用 xeCJK 以便更好控制
\usepackage{pgffor}
\usepackage{ragged2e}
\usepackage{booktabs}  % 表格支持
\usepackage{amsmath}   % 公式支持
\usepackage{hyperref}  % 目录跳转

\geometry{top=2.5cm,bottom=2.5cm,left=3cm,right=3cm}

% ------- 字体配置 (适配本地文件) ------- 
% 使用 Fandol 作为基础中文字体(Tectonic 自动下载)
\setCJKmainfont[
    BoldFont=FandolSong-Bold.otf,
    ItalicFont=FandolKai-Regular.otf
]{FandolSong-Regular.otf}

% 定义华文行楷 (使用当前目录下的文件)
% AutoFakeBold 可以让它支持伪粗体
\newCJKfontfamily\xingkai[Path=./, AutoFakeBold]{STXingkai.TTF}

% ------- 变量定义 (来自 Jinja2) ------- 
% 我们不需要 \newcommand 来定义变量了,直接在正文中插入 {{ variable }}

% 封面用横线填写区
\newcommand{\fillin}[1]{\underline{\makebox[7cm][l]{\hspace{0.5cm}#1}}}
\newcommand{\fillinshort}[1]{\underline{\makebox[1cm][c]{ #1}}}

% ------- 正文开始 ------- 
\begin{document}

% 封面
\begin{titlepage}
    \centering
    % 确保 logo 存在,否则用占位符
    \IfFileExists{school_logo.png}{
        \includegraphics[width=10cm]{school_logo.png}
    }{
        \vspace{2cm} \textbf{[School Logo Missing]} \vspace{1cm}
    }

    \vspace{2.5cm}

    % 字号设置:TeX 的 \zihao 命令通常由 ctex 提供。
    % 这里我们用标准字号命令模拟,或者如果用 ctexart 类则直接可用。
    % 为了保持对你原模版的最大兼容,我们改用 ctexart 类,或者手动定义 zihao
    % 简单起见,这里用 fontsize 模拟:
    % \zihao{-0} 约等于 36pt (小初)
    % \zihao{-2} 约等于 22pt (小二)
    % \zihao{3}  约等于 16pt (三号)
    % \zihao{4}  约等于 14pt (四号)
    % \zihao{-4} 约等于 12pt (小四)
    
    {\fontsize{36pt}{40pt}\selectfont \xingkai \textbf{《{{ e(title) }}》}}\\[1cm]
    {\fontsize{36pt}{40pt}\selectfont \xingkai \textbf{课程论文}}\\[3cm]

    {\fontsize{22pt}{30pt}\selectfont
    \begin{tabular}{p{5.3cm}p{10cm}}
        \raggedleft \textbf{学\qquad 院}: & \fillin{ {{ e(college) }} } \\[0.4cm]
        \raggedleft \textbf{专\qquad 业}: & \fillin{ {{ e(major) }} } \\[0.4cm]
        \raggedleft \textbf{年\qquad 级}: & \fillin{ {{ e(grade) }} } \\[0.4cm]
        \raggedleft \textbf{学\qquad 号}: & \fillin{ {{ e(student_id) }} } \\[0.4cm]
        \raggedleft \textbf{姓\qquad 名}: & \fillin{ {{ e(student_name) }} } \\[0.4cm]
        \raggedleft \textbf{指导老师}: & \fillin{ {{ e(supervisor) }} } \\ 
    \end{tabular}
    }

    \vfill

    {\fontsize{14pt}{20pt}\selectfont
    完成日期:\quad \fillinshort{ {{ e(finish_year) }} }\quad 年\quad \fillinshort{ {{ e(finish_month) }} }\quad 月
    }
\end{titlepage}

% 扉页(中文英文摘要)
\newpage
\pagenumbering{roman}

\centering
{\fontsize{16pt}{20pt}\selectfont \textbf{ {{ e(title) }} }}\\[0.5cm]
\vspace{0.5cm}
{\fontsize{14pt}{18pt}\selectfont
{{ e(student_name) }}\qquad 指导老师:{{ e(supervisor) }}\\[1cm]
}

\noindent
{\fontsize{16pt}{20pt}\selectfont \textbf{摘\quad 要}}\\[0.5cm]

\justifying 
\noindent
{\fontsize{12pt}{18pt}\selectfont
{{ e(abstract_zh) }}\\[1cm]
}

\noindent
\textbf{关键词:{{ e(keywords_zh) }}}

\newpage
\centering
{\fontsize{16pt}{20pt}\selectfont \textbf{ {{ e(title_en) }} }}\\[0.5cm]
\vspace{0.5cm}
{\fontsize{14pt}{18pt}\selectfont
{{ e(student_name_en) }}\qquad Supervisor: {{ e(supervisor_en) }}\\[1cm]
}

\noindent
{\fontsize{16pt}{20pt}\selectfont \textbf{Abstract}}\\[0.5cm]

\justifying 
\noindent
{\fontsize{12pt}{18pt}\selectfont
{{ e(abstract_en) }}\\[1cm]
}

\noindent
\textbf{Keywords: {{ e(keywords_en) }}}

% 目录页
\newpage
\centering
\vspace{1cm}
\tableofcontents

% 正文开始
\newpage
\pagenumbering{arabic}
\setcounter{page}{1}

\justifying 

% 动态生成章节


    
        \section{ {{- e(block.title) -}} }
        % section 自动会进目录,不需要手动 addcontentsline

    
        \subsection{ {{- e(block.title) -}} }

    
        \par {{ e(block.content) }}
        \vspace{0.5em}

    
        \begin{equation}
            {{ block.content }}
        \end{equation}

    
        \begin{figure}[h]
            \centering
            \includegraphics[width=0.8\textwidth]{ {{- block.path -}} }
            \caption{ {{- e(block.caption) -}} }
        \end{figure}

    
        \begin{table}[h]
            \centering
            \caption{数据表}
            \vspace{0.2em}
            \begin{tabular}{ {{ 'c' * block.headers|length }} }
                \toprule
                
                    \textbf{ {{- e(h) -}} }  & 
                 \\ 
                \midrule
                
                    
                        {{ e(cell) }}  & 
                     \\ 
                
                \bottomrule
            \end{tabular}
        \end{table}

    



\end{document}
