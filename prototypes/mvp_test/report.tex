\documentclass[12pt,a4paper]{article}
\usepackage{graphicx}
\usepackage{geometry}
\usepackage{fontspec}
\usepackage{xeCJK}     % 显式使用 xeCJK 以便更好控制
\usepackage{pgffor}
\usepackage{ragged2e}
\usepackage{booktabs}  % 表格支持
\usepackage{amsmath}   % 公式支持
\usepackage{hyperref}  % 目录跳转

\geometry{top=2.5cm,bottom=2.5cm,left=3cm,right=3cm}

% ------- 字体配置 (适配本地文件) ------- 
% 使用 Fandol 作为基础中文字体(Tectonic 自动下载)
\setCJKmainfont[
    BoldFont=FandolSong-Bold.otf,
    ItalicFont=FandolKai-Regular.otf
]{FandolSong-Regular.otf}

% 定义华文行楷 (使用当前目录下的文件)
% AutoFakeBold 可以让它支持伪粗体
\newCJKfontfamily\xingkai[Path=./, AutoFakeBold]{STXingkai.TTF}

% ------- 变量定义 (来自 Jinja2) ------- 
% 我们不需要 \newcommand 来定义变量了,直接在正文中插入 

% 封面用横线填写区
\newcommand{\fillin}[1]{\underline{\makebox[7cm][l]{\hspace{0.5cm}#1}}}
\newcommand{\fillinshort}[1]{\underline{\makebox[1cm][c]{ #1}}}

% ------- 正文开始 ------- 
\begin{document}

% 封面
\begin{titlepage}
    \centering
    % 确保 logo 存在,否则用占位符
    \IfFileExists{school_logo.png}{
        \includegraphics[width=10cm]{school_logo.png}
    }{
        \vspace{2cm} \textbf{[School Logo Missing]} \vspace{1cm}
    }

    \vspace{2.5cm}

    % 字号设置:TeX 的 \zihao 命令通常由 ctex 提供。
    % 这里我们用标准字号命令模拟,或者如果用 ctexart 类则直接可用。
    % 为了保持对你原模版的最大兼容,我们改用 ctexart 类,或者手动定义 zihao
    % 简单起见,这里用 fontsize 模拟:
    % \zihao{-0} 约等于 36pt (小初)
    % \zihao{-2} 约等于 22pt (小二)
    % \zihao{3}  约等于 16pt (三号)
    % \zihao{4}  约等于 14pt (四号)
    % \zihao{-4} 约等于 12pt (小四)
    
    {\fontsize{36pt}{40pt}\selectfont \xingkai \textbf{《基于深度学习的图像识别研究》}}\\[1cm]
    {\fontsize{36pt}{40pt}\selectfont \xingkai \textbf{课程论文}}\\[3cm]

    {\fontsize{22pt}{30pt}\selectfont
    \begin{tabular}{p{5.3cm}p{10cm}}
        \raggedleft \textbf{学\qquad 院}: & \fillin{ 计算机科学与工程学院 } \\[0.4cm]
        \raggedleft \textbf{专\qquad 业}: & \fillin{ 软件工程 } \\[0.4cm]
        \raggedleft \textbf{年\qquad 级}: & \fillin{ 2023级 } \\[0.4cm]
        \raggedleft \textbf{学\qquad 号}: & \fillin{ 20230001 } \\[0.4cm]
        \raggedleft \textbf{姓\qquad 名}: & \fillin{ 张小明 } \\[0.4cm]
        \raggedleft \textbf{指导老师}: & \fillin{ 李教授 } \\ 
    \end{tabular}
    }

    \vfill

    {\fontsize{14pt}{20pt}\selectfont
    完成日期:\quad \fillinshort{ 2026 }\quad 年\quad \fillinshort{ 6 }\quad 月
    }
\end{titlepage}

% 扉页(中文英文摘要)
\newpage
\pagenumbering{roman}

\centering
{\fontsize{16pt}{20pt}\selectfont \textbf{ 基于深度学习的图像识别研究 }}\\[0.5cm]
\vspace{0.5cm}
{\fontsize{14pt}{18pt}\selectfont
张小明\qquad 指导老师:李教授\\[1cm]
}

\noindent
{\fontsize{16pt}{20pt}\selectfont \textbf{摘\quad 要}}\\[0.5cm]

\justifying 
\noindent
{\fontsize{12pt}{18pt}\selectfont
本文提出了一种改进的卷积神经网络模型,用于解决复杂背景下的图像识别问题。实验结果表明,该模型在CIFAR-10数据集上的准确率达到了95\%以上,显著优于传统方法。\\[1cm]
}

\noindent
\textbf{关键词:深度学习;图像识别;卷积神经网络}

\newpage
\centering
{\fontsize{16pt}{20pt}\selectfont \textbf{ Research on Image Recognition Based on Deep Learning }}\\[0.5cm]
\vspace{0.5cm}
{\fontsize{14pt}{18pt}\selectfont
Xiaoming Zhang\qquad Supervisor: Prof. Li\\[1cm]
}

\noindent
{\fontsize{16pt}{20pt}\selectfont \textbf{Abstract}}\\[0.5cm]

\justifying 
\noindent
{\fontsize{12pt}{18pt}\selectfont
This paper proposes an improved Convolutional Neural Network (CNN) model to solve the problem of image recognition in complex backgrounds. The experimental results show that the accuracy of the model on the CIFAR-10 dataset reaches more than 95\%, which is significantly better than traditional methods.\\[1cm]
}

\noindent
\textbf{Keywords: Deep Learning; Image Recognition; CNN}

% 目录页
\newpage
\centering
\vspace{1cm}
\tableofcontents

% 正文开始
\newpage
\pagenumbering{arabic}
\setcounter{page}{1}

\justifying 

% 动态生成章节


    
        \section{引言}
        % section 自动会进目录,不需要手动 addcontentsline

    



    
        \par 随着人工智能技术的飞速发展,计算机视觉已成为最热门的研究领域之一。
        \vspace{0.5em}

    



    
        \section{相关工作}
        % section 自动会进目录,不需要手动 addcontentsline

    



    
        \par 近年来,许多研究者提出了各种基于CNN的改进模型。例如,ResNet通过引入残差连接解决了深层网络的梯度消失问题。
        \vspace{0.5em}

    



    
        \section{实验结果}
        % section 自动会进目录,不需要手动 addcontentsline

    



    
        \begin{table}[h]
            \centering
            \caption{数据表}
            \vspace{0.2em}
            \begin{tabular}{ cccc }
                \toprule
                \textbf{模型}  & \textbf{Top-1 准确率 (\%)}  & \textbf{Top-5 准确率 (\%)}  & \textbf{参数量 (M)}  \\ 
                \midrule
                
                    ResNet-18  & 69.76  & 89.08  & 11.7  \\ 
                
                    ResNet-50  & 76.15  & 92.87  & 25.6  \\ 
                
                    Ours  & 78.20  & 94.10  & 20.1  \\ 
                
                \bottomrule
            \end{tabular}
        \end{table}

    



    
        \section{结论}
        % section 自动会进目录,不需要手动 addcontentsline

    



    
        \par 本文提出的方法在保证计算效率的同时,有效提升了识别准确率。
        \vspace{0.5em}

    



\end{document}